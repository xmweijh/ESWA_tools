%% 
%% Copyright 2019-2020 Elsevier Ltd
%% 
%% This file is part of the 'CAS Bundle'.
%% --------------------------------------
%% 
%% It may be distributed under the conditions of the LaTeX Project Public
%% License, either version 1.2 of this license or (at your option) any
%% later version.  The latest version of this license is in
%%    http://www.latex-project.org/lppl.txt
%% and version 1.2 or later is part of all distributions of LaTeX
%% version 1999/12/01 or later.
%% 
%% The list of all files belonging to the 'CAS Bundle' is
%% given in the file `manifest.txt'.
%% 
%% Template article for cas-dc documentclass for 
%% double column output.

%\documentclass[a4paper,fleqn,longmktitle]{cas-dc}
\documentclass[a4paper,fleqn]{cas-dc}

% \usepackage[authoryear,longnamesfirst]{natbib}
\usepackage[authoryear]{natbib}
% \usepackage[numbers,sort]{natbib}
\usepackage{hyperref}
\usepackage{graphicx}
\usepackage{subfigure}
\usepackage{epstopdf}
\usepackage{color}
\usepackage[noend]{algpseudocode}
\usepackage{algorithmicx,algorithm}

%%%Author definitions
\def\tsc#1{\csdef{#1}{\textsc{\lowercase{#1}}\xspace}}
\tsc{WGM}
\tsc{QE}
\tsc{EP}
\tsc{PMS}
\tsc{BEC}
\tsc{DE}
%%%

\begin{document}
\begin{sloppypar}
	\let\WriteBookmarks\relax
	\def\floatpagepagefraction{1}
	\def\textpagefraction{.001}
	\let\printorcid\relax
	\shorttitle{}
	\shortauthors{S. Zhang et~al.} %% 缩略作者 自己名字, 比如: 张三 = S. Zhang

	%% 标题
	\title [mode = title]{Community Detection}
	%%\tnotemark[1,2]

	%%\tnotetext[1]{This document is the results of the research project funded by the National Science Foundation.}

	%%\tnotetext[2]{The second title footnote which is a longer text matter to fill through the whole text width and overflow into another line in the footnotes area of the first page.}


	%% 作者顺序
	%% 1
	\author[1]{\textcolor[RGB]{0,0,1}{Zhangsan}}
	\fnmark[1]
	% \cormark[1]%%通讯作者星标
	\ead{abc@123.edu.cn}


	%% 2
	\author[1]{\textcolor[RGB]{0,0,1}{Lisi}}
	\fnmark[1] %%第几作者
	%\credit{}%%本文的贡献
	\address[1]{School of Computer Science and Technology, ***, Chongqing 400065, China}
	\cormark[1]%%通讯作者星标
	\ead{abc@123.edu.cn}

	\author[1]{\textcolor[RGB]{0,0,1}{Wangermazi}}
	\ead{abc@123.edu.cn}




	\cortext[cor1]{Corresponding author.} %% 首页左下角通讯作者
	%%\cortext[cor2]{Principal corresponding author} 

	\fntext[fn1]{Equal Contribution.}
	%%\fntext[fn2]{Another author footnote, this is a very long footnote and it should be a really long footnote. But this footnote is not yet sufficiently long enough to make two lines of footnote text.}

	%%\nonumnote{This note has no numbers. In this work we demonstrate $a_b$ the formation Y\_1 of a new type of polariton on the interface between a cuprous oxide slab and a polystyrene micro-sphere placed on the slab.}

	%%摘要
	\begin{abstract}
		Community detection 
	\end{abstract}

	% \begin{graphicalabstract}
	% 	%%\includegraphics{./grabs.pdf} %%图片摘要地址路径
	% \end{graphicalabstract}

	%%高亮
	% \begin{highlights}
	% 	\item End-to-end community detection method based on graph convolution network.
	% 	\item A new community perspective similarity is proposed.
	% 	\item Modify the convolution layer for large networks.
	% 	\item The loss function based on modularity and Bernoulli Poisson model is introduced.
	% 	\item Evaluate performance using real-world networks.
	% \end{highlights}
		
	%% 关键词
	\begin{keywords}
		Community detection \sep
		Graph convolutional network 
	\end{keywords}

	% 此指令为生成标题格式,不可删除
	\maketitle

	%% 1.引言
	\section{Introduction}

	%%\par{文本内容}换行并缩进
	\par{
		Many approaches have been proposed.
	}

	\par{
		Traditional methods \citep{9732192}.
	}



	\begin{enumerate}[(1)]
		\item We 
		\item networks.
		\item Extensive experiments
	\end{enumerate}

	\par{
		The rest of this paper is organized as follows. In Section \ref{Related_work}.
	}


	%% 图1  后面要加就自己复制这个改改
	\begin{figure}[h] % htbp详见说明书,记得删除括号内容
		\centering % 居中
		\includegraphics[width=0.9\linewidth]{./123.png}% 图片地址,可以pdf可以jpg,scale是缩放比例
		\caption{buxiangkanlunwenle.} % 图片标题
		\label{FIG:1} % 这里只要改冒号后面的数字,图片几就是几
	\end{figure}


	%% 2.第二章
	\section{Related work}
	\label{Related_work}
	\par{
		Our work
	}

	\subsection{Community detection}
	\par{
		Community detection 
	}

	\subsection{Graph}

	\par{
		Graph
	}



	\section{Preliminaries}
	\label{Preliminaries}
	\par{
		In this section
	}


	\section{Methodology}
	\label{Methodology}
	\par{
		This section 
	}


	\begin{algorithm}[ht]
		\caption{123}
		\label{alg:1}
		\hspace*{0.02in} {\bf Input:}
		123.\\
		\hspace*{0.02in} {\bf Output:}
		123
		\begin{algorithmic}[1]
			\State 123
			\State 123
			\State 123
			\For{$t=1...MaxIter$}
			\State 123
			\State 123
			\State 123
			\If{123}
			\State 123
			\EndIf
			\If{$early\_stopping > 100 $}
			\State Break.
			\EndIf
			\State Update the parameters in the model with the Adam optimizer.
			\EndFor
			\State \Return 123
		\end{algorithmic}
	\end{algorithm}


	\section{EXPERIMENTS}
	\label{EXPERIMENTS}
	\par{
		In this section
	}

	\begin{table}[width=.9\linewidth,cols=4,pos=ht]
		\caption{Datasets for Community Detection}
		\label{tbl1}
		\begin{tabular}{lllll}
			\hline
			\textbf{Dataset} & \multicolumn{1}{c}{\textbf{N}} & \multicolumn{1}{c}{\textbf{M}} & \multicolumn{1}{c}{\textbf{K}} & \multicolumn{1}{c}{\textbf{|X|}} \\ \hline
			CiteSeer         & 3327                           & 4552                           & 6                              & 3703                             \\
			CiteSeer-full    & 4230                           & 5337                           & 6                              & 602                              \\
			Cora             & 2708                           & 5278                           & 7                              & 1433                             \\
			Cora-ML          & 2995                           & 8158                           & 7                              & 2879                             \\
			Cora-full        & 19793                          & 63421                          & 70                             & 8710                             \\
		\end{tabular}
	\end{table}


	\section{Conclusion}
	\label{Conclusion}
	\par{
		This paper
	}

	\section*{CRediT authorship contribution statement}
	\par{Zhangsan: Methodology, Conceptualization, Investigation, Writing - Review \& Editing. }

	\section*{Declaration of competing interest}
	\par{The authors declare that they have no known competing financial interests or personal relationships that could have appeared to influence the work reported in this paper.}

	\section*{Funding}
	\par{This work was jointly supported by the following projects: }

	% 致谢
	\section*{Acknowledgements}
	\par{The authors would like to thank the reviewers for their insightful comments and useful suggestions.}



	%% Loading bibliography style file
	\bibliographystyle{model5-names}
	% \bibliographystyle{cas-model2-names}

	% Loading bibliography database
	\bibliography{cas-refs}

	%\vskip3pt

	%\bio{}
	%Author biography without author photo.

	%\endbio

\end{sloppypar}
\end{document}

